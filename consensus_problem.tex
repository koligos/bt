
\def\ctustyle{{\tenss CTUstyle}}
\def\ttb{\tt\char`\\} % pro tisk kontrolních sekvencí v tabulkách

\chap Distributed algorithms


A very good and step-by-step introduction to the theory of Distributed algorithms may be found in \cite[Raynal2013].

% main topic linear consensus algorithm - from springer - algorithms in communication meesage systems 
%what's distributed, synchronous; reliable channel


\chap Linear  average consensus algorithm

In this chapter, let us consider an undirected and connected graph $G=(V,E)$ with $N$ vertices and edges $(v_i,v_j)$ between vertices  $i, j$, where  $i,j \in \{1,2, ..., N\}$. We denote an initial value $x_i(0)$ the value assigned to the $i$-th vertex (node, agent) in time $0.$ Then  $x_i(t)$ refers to the value in the $i$-th vertex in time $t$. Our goal is to, for $t \rightarrow \infty$, using local communication and computation, in all  $N$ vertices of the graph,  obtain an average value of all these initial values. Based on a matrix-like description of graph $G$, our goal will be to construct matrix ${\bi Q}$, whose components $q_{ij}$ will in fact bear this averaging algorithm, in a formalism of iterative matrix multiplication. 

In this chapter, subject of our interest will be a linear, discrete-time consensus algorithm. A detailed description of the folowing is in \cite[Garin2010], and it contains also rich references to other publications.

\sec Introduction



Assume this {\em linear} update equation 
$$ x(t+1) = {\bi Q}(t) x(t), $$
where $x(t)= ( x_1(t), x_2(t), ..., x_N(t))^T \in  {\bbchar R}^{N} $ 
and for all values of $t$, $  {\bi Q} (t) \in  {\bbchar R}^{N{\times{N}}}  $ is a {\em stochastic matrix}, i.e. $q_{ij} (t) \ge 0$ and $\sum _{j=1}^N q_{ij} = 1, \forall i ,j \in {1, 2, ..., N}.$ Meaning, that all values in each row sum up to 1. The $q_{ij}$ components are also often reffered to as {\em weights.}

Now, let's  rewrite the equation above expanding a matrix multiplication:

$$ x(t+1) = \sum_{j=1}^N q_{ij}(t) x_i (t) = x_i(t) + \sum_{j=1; j\not= i}^N q_{ij}(x_j(t)-x_i(t)).$$
This equation is for given ${\bi Q} (t)$ a general form of a {\em linear consensus algorithm}, that may be usually found in the literature. Frankly spoken, all the theory behind linear consensus algorithm aims to find the best matrix $  {\bi Q}(t)$, such as the consensus is reached. 

Formally defined, we say that $  {\bi Q}(t)$ solves {\em consensus problem}, if for all $i$ holds  $$\lim_{t\rightarrow\infty} x_i (t)= \alpha, \forall i.$$ Then, for a solution of the {\em average consensus problem} must be in an addition to the previous condition fulfilled also $$\alpha = {1\over N} \sum_{i=1}^Nx_i(0).$$


Moreover, we call   $  {\bi Q}(t)$ {\em doubly stochastic}, if holds also $\sum _{j=1^N} q_{ij} = 1, \forall j \in {1, 2, ..., N}.$ So both, rows and columns sum up to 1. Note, that  if $  {\bi Q} (t) $ is stochastic and symetric, $  {\bi Q}(t) = {\bi Q}(t) ^T$, then $  {\bi Q}(t)$ is doubly stochastic.

The ${\bi Q} (t)$ matrix may be considered as: 1) constant ${\bi Q} (t) = {\bi Q}$, 2) a deterministic time variable matrix, 3) randomly variable matrix. For simplicity, we will firsly concern only case 1). %The are behind scope of this text.  


Now, let's have a look at a very useful theorem, that originaly comes from closely related topic of Markov Chains.

\Theorem (From \cite[Garin2010].)
Let us consider the sequence of constant matrices ${\bi Q} (t) = Q$. If the
graph G Q ∈ G sl and is rooted, then Q solves the consensus problem, and
lim Q t = 1 η T
t→∞
where η ∈ R N is the left eigenvector of Q for the eigenvalue one and has the prop-
erties η i ≥ 0 and 1 T η = 1. If G Q is strongly connected, then η i > 0, ∀i. If in addition
Q is doubly-stochastic, then G Q is strongly connected and Q solves the average con-
sensus problem, i.e. η = N 1 1. Moreover, in all cases the convergence is exponential
and its rate is given by the essential spectral radius esr(Q).

\sec Motivation

Let's think about an experiment, where few nodes aim to provide only one result of measurement based on many local measurments. For example we measure an average temperature in a room. Very acurate measure devices are expensive. We can generally try to replace small number of very good devices by some probably bigger number of less reliable devices whose benefit will be an interchange of information between near nodes.

 They  We want to replace a number of nodes, that exchange informatMany less accurate a reliable nodes as an alternative to very accurate and very reliable but also very expensive nodes.


\sec Discrete and continuous time

\sec Convergenece analysis


asdsafsaf
