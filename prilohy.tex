
\bibchap
\ifx\usebib\undefined
   \usebbl/c mybase
\else
   \usebib/c (simple) mybase
\fi



\app Matlab scripts

\label[perfect_script]
\sec Matlab script: Average consensus algorithm with perfect communication

This is script $perfect\_communication\_example.m$ used in Examples \ref[ex3_5], \ref[ex3_6] and \ref[ex3_7]  to present the run of algorithm. In this case we do not use the built-in matlab functions to compute desired values, only to check them.

%\picw=0.7\hsize % obrázek na šířku sazby
%\cinspic scripts_export/script_perfect_communication_example.pdf

\begtt
% Run of average consensus algorithm with perfect communication       
%###########################################################            
clear all;
s=[1 1 2 2 3 3 3 4 4 4 5 6 7 8 8 9 9 10  ];%source vertices
t=[2 3 4 5 7 10 5 5 6 7 6 7 5 1 2 10 7 8];%destination vertices
w=[1 1 1 1 1 1 1 1 1 1 1 1 1 1 1 1 1 1]; %weights of edges
G = graph(s,t,w); %create graph G with parameters s, t, w
iterations = 30;
for i= 1:length(s) %A is adjacency matrix; A(i,j)=1 <=> exists edge (i,j)
   A(s(i),t(i))=1;
   A(t(i),s(i))=1; 
end
checkA=isequal(A, adjacency(G)); %check adjacency matrix
M=0; %M is the highest degree of vertex
for i=1:size(A)
    sumRadek=sum(A(i,:))
    if sumRadek>=M
        M=sumRadek;        
    end
end
for i = 1:size(A)
    D(i,i)=sum(A(i,:))%D is degree matrix
end
L=D-A; %Laplacian matrix
checkLaplacian=isequal(laplacian(G),L); %check Laplacian 
for i= 1:length(s) %Incidence matrix
    E(s(i),i)=1;
    E(t(i),i)=-1;
end
L2=E*transpose(E); %another way to compute Laplacian
checkLaplacian2=isequal(laplacian(G),L2); %check Laplacian 
I=L*ones(max(s)); %check that Laplacian rows sum up to 1
D = eig(L,'matrix');%diagonal matrix of eigenvalues of laplacian
L_eig=eig(L);
alpha=2/(L_eig(2)+L_eig(max(s)))

for i = 1:max(s)
node_initial_value(i)=i; %initialization of values to average
%running_value(i+1,j)=node_value(j);
running_value(1,i)=node_initial_value(i);
running_value_Error(1,i)=-node_initial_value(i)+mean(node_initial_value);
end
for i= 1:max(s)
final_value(i)=mean(node_initial_value); %expected average value
end
node_value=node_initial_value; %inicialization of nodes values
running_value_Error(1,:)=final_value-node_initial_value;
for i=1:iterations+1
discrete_time(i)= i-1;
end
for i = 1:iterations;
    IT=eye([max(s) max(s)])-alpha*L; %iteratation Perron matrix
    node_value=node_value;
    node_value= node_value* IT;
    for j=1:max(s)
            running_value(i+1,j)=node_value(j);
            running_value_Error(i+1,j)=final_value(j)-node_value(j);
    %node_value(5)=3; uncomment for convergence to v_3 initial value
    end
end

figure;
plot(discrete_time, running_value(:,:));
ttl1=title( 'Run of algorithm with updates affected by AWGN '  )
ttl1=set(ttl1,'Interpreter','latex','FontSize', 15);
xlbl1=xlabel('Iterations [-]');
xlbl1=set(xlbl1,'Interpreter','latex');
ylbl1=ylabel('Value [-]');
ylbl1=set(ylbl1,'Interpreter','latex');
grid;
figure;
grid on
plot(discrete_time, running_value_Error(:,:));
title('shrinking of error');
grid on
figure;
plot(G); %plot graph G
\endtt

\nextoddpage

\label[noise_script]
\sec Matlab script: Average consensus algorithm with noisy observation and updates using decreasing step size

 This is script $script\_noisy\_updates\_gamma.m$ used in Example \label[ex_conv_noise_gamma] to present simple way to preserve convergence in case of noisy updates.

\begtt
% Run of average consensus algorithm with noisy 
% updates using descending step size scheme gamma=a/(1+b)^c                     


%% graph definition
clear all; clc;
variance_noise=0.1; %variance of the gaussian noise added to the updates
iterations =5000; %number of iterations of the algorithm
nodes=60; %number of graph nodes
G=graph(bucky); % using matlab default graph bucky
% i.e. 60 nodes with degree 3
%% variables initialization
equiv_noise_vector=zeros(1,nodes); 
x_0=zeros(1,nodes);
MSE_ave=zeros(zeros(1,iterations+1));
VAR=zeros(zeros(1,iterations+1));
MSE_act=zeros(zeros(1,iterations+1));
diag_D=zeros(nodes,nodes);
%plot the used graph
figure;
plot(G)
deg=degree(G);
for p=1:nodes
   diag_D(p,p)=deg(p); 
end
%observation initialization const 10 + wgn with variance 1 
for i=1:nodes     %initialize measurement
    x_0(i)=(10+wgn(1,1,log10(10*variance_noise)));    
end

%% variables definition
x_est_run(1,:) = x_0;
MSE_act(1)= sum((x_0-mean(x_0)).^2);
MSE_ave(1)=MSE_act(1);
VAR(1)=var(x_0);
noise_matrix=zeros(nodes);
%Laplacian using library functions
A=adjacency(G); %adjacency matrix
L=laplacian(G); %laplacian matrix
eig_L=eig(L);%eigenvalues of Laplacian
gamma=2/(eig_L(2)+eig_L(nodes));%the best possible coefficient
mean_0=mean(x_0);%value to converge to

%% Iterations of the algorithm
for i=1:iterations
   gamma=1/(i+42)^0.75; %selected descending step size
   P=eye(nodes)- gamma* L; %Perron matrix

   for j=1:nodes %computation of the nise affecting updates exchange
      noise_matrix(j,:)=wgn(nodes,1,10*log10(variance_noise)) ;
      noise_matrix(j,j)=0 ;
   end
    equiv_noise_matrix= P*noise_matrix;
   for j=1:nodes
     equiv_noise_vector(j)= equiv_noise_matrix(j,j);
   end
   x_est_run(i+1,:)=  P*x_est_run(i,:)'+equiv_noise_vector';  
end

%calculation of the run statistics
for i=1:iterations
  MSE_ave(i+1)= sum((x_est_run(i+1,:)-mean(x_0)).^2)/nodes;
  MSE_act(i+1)= sum((x_est_run(i+1,:)-mean(x_est_run(i+1,:))).^2)/nodes;   
  VAR(i+1)= var(x_est_run(i+1,:));
end

%%  Plot results
figure;
plot(0:iterations, x_est_run(:,:));
ttl1=title( 'Values in nodes'  )
ttl1=set(ttl1,'Interpreter','latex','FontSize', 15);
xlbl1=xlabel('Iterations [-]');
xlbl1=set(xlbl1,'Interpreter','latex');
ylbl1=ylabel('Value [-]');
ylbl1=set(ylbl1,'Interpreter','latex');
grid;

figure;
loglog(0:iterations, MSE_ave(:)   );
ttl1=title( 'Mean square error w.r.t. initial average '  )
ttl1=set(ttl1,'Interpreter','latex','FontSize', 15);
xlbl1=xlabel('Iterations [-]');
xlbl1=set(xlbl1,'Interpreter','latex');
ylbl1=ylabel('MSE [-]');
ylbl1=set(ylbl1,'Interpreter','latex');
grid;

figure;
loglog(0:iterations, VAR(:));
ttl1=title( 'Variance '  );
ttl1=set(ttl1,'Interpreter','latex','FontSize', 15);
xlbl1=xlabel('Iterations [-]');
xlbl1=set(xlbl1,'Interpreter','latex');
ylbl1=ylabel('Variance [-]');
ylbl1=set(ylbl1,'Interpreter','latex');
grid;

\endtt




\label[nodes_number_script]
\sec Matlab script: Estimation of nodes number in given area



\begtt
% Script to find number of N sensors based on their ID
%###########################################################

%% Initialization
clear all;
nodes = 99;
iterations = 30;
G = WattsStrogatz(nodes,4,0.5); %generates random graph
L=full(laplacian(G));
degreeMax= max(max(L));
alpha=1/degreeMax;
P=eye(nodes)- alpha* L; %Perron matrix

for i = 1:nodes
x_running(1,i)=i; %initialization vertex with its ID
end

%% iterations of the algorithm
for i = 1:iterations
    x_running(1+i,:)=P*x_running(i,:)';
end

%% plot results
figure;
plot(0:iterations, x_running(:,:));
ttl1=title( 'Average ID number estimation'  )
ttl1=set(ttl1,'Interpreter','latex','FontSize', 15);
xlbl1=xlabel('Iterations [-]');
xlbl1=set(xlbl1,'Interpreter','latex');
ylbl1=ylabel('Node ID [-]');
ylbl1=set(ylbl1,'Interpreter','latex');
grid;

figure;
h = plot(G);
c = h.EdgeColor;
h.EdgeColor = 'k';
h.LineWidth=1
h.MarkerSize=8;
h.EdgeAlpha=1;
labelnode(h,1:nodes,' ')

\endtt

\label[dynamic_tracking]
\sec Matlab script: Tracking of dynamic target



\begtt

%Tracking of dynamic target script
%###########################################################
clear all;
nodes = 5; % 4 followers and 1 target
d=0.01;%distance between followers
r=200; % range of plot axis (xy)
iterations = 1000; 
s=[1 1 1 1 2 2 2 3 3 4 ];%graph definition
t=[5 2 3 4 5 3 4 5 4 5];
v1_pos=[100,100]; %initialize followers position
v2_pos=[-100,100];
v3_pos=[100,-100];
v4_pos=[-100,-100];
v5_init_pos=[100*rand, 100*rand]; %target position is random

G =graph(s,t)%generates  graph
L=full(laplacian(G)); %matlab function for Laplacian
degreeMax= max(max(L));%generate weight that control dynamics
alpha=0.08*1/degreeMax; %coefficient 0.08 found experimentaly
P=eye(nodes)- alpha* L; %Perron matrix

x_pos_vect(1,:)=[v1_pos(1) v2_pos(1) v3_pos(1) v4_pos(1) v5_init_pos(1)];
y_pos_vect(1,:)=[v1_pos(2) v2_pos(2) v3_pos(2) v4_pos(2) v5_init_pos(2)];

v5_pos_x(1)=v5_init_pos(1);
v5_pos_y(1)=v5_init_pos(2);


for i = 1:iterations
 x_pos_vect(i+1,:)=P* x_pos_vect(i,:)';
y_position_vector(i+1,:)=P*y_position_vector(i,:)';

if i<0.5*iterations %for first half of iterations the target moves
y_position_vector(i+1,5)=y_position_vector(i,5)+0.5*rand*(-1^(i));
 x_pos_vect(i+1,5)= x_pos_vect(i,5)+0.5*rand*(-1^(i));
end

% distance between followers
y_position_vector(i+1,1)=y_position_vector(i+1,1)+d;
 x_pos_vect(i+1,1)= x_pos_vect(i+1,1)+d;

y_position_vector(i+1,2)=y_position_vector(i+1,1)+d;
 x_pos_vect(i+1,2)= x_pos_vect(i+1,5)-d;

y_position_vector(i+1,3)=y_position_vector(i+1,3)-d;
 x_pos_vect(i+1,3)= x_pos_vect(i+1,3)+d;

y_position_vector(i+1,4)=y_position_vector(i+1,4)-d;
 x_pos_vect(i+1,4)= x_pos_vect(i+1,4)-d;

end

%% plot the animation of mooving point
figure('Position', [100, 100, 450,300])
for i=1:iterations
hold off;
plot(x_pos_vect(i,5),y_pos_vect(i,5),'ob','MarkerSize',8)
hold on;
plot(x_pos_vect(i,1),y_pos_vect(i,1),'or','MarkerFaceColor','r') 
hold on;
plot(x_pos_vect(i,2),y_pos_vect(i,2),'or',''MarkerFaceColor','r') 
hold on;
plot(x_pos_vect(i,3),y_pos_vect(i,3),'or','MarkerFaceColor','r') 
hold on;
plot(x_pos_vect(i,4),y_pos_vect(i,4),'or','MarkerFaceColor','r') 
hold on;
grid;
 axis([-r r -r r])
 pause(0.001)
end
\endtt

\label[time_sync_app]
\sec Matlab script: (Measurement) Time base synchronization

This is script used to simulate synchronization of Time base, meant to e.g. synchronize the sensors to do the measurement at one common instant. See Section \ref[distr_time_sync].

\begtt
% Time base synchronization script
%###########################################################
%% Initialization
clear all;
nodes = 30; %number of nodes of graph
iterations = 150; %number of iterations of the algorithm
G = WattsStrogatz(nodes,2,1); %generates random graph
L=full(laplacian(G)); %Laplacian using Matlab function
degreeMax= max(max(L)); %find the greatest degree of graph
alpha=1/degreeMax; %choose parameter to Define Perron matrix
P=eye(nodes)- alpha* L; %create Perron matrix
commonTimeBase=1000; % some big number used to initialize time base
% e.g. in gps time used number of weeks and secs since January'80

for i = 1:nodes %initialize the vertices time base + some random offset
time_running(1,i)=commonTimeBase+1000*rand; %init of time base + offset
end
for i = 1:iterations
    %average consensus iterations 
    time_running(1+i,:)=P*time_running(i,:)'; 
    %each vertex adds to the computed value some addition,... 
    %... representing ticks of its internal oscilator
    time_running(1+i,:)= time_running(1+i,:)+1+2*rand;
    % between 300th and 350th we simulate an outage of 10 sensors
    if (i>300)&(i<500)
    time_running(1+i,10:20)= time_running(200,20:30)   ;
    end
end

%% plot results
figure;
plot(0:iterations, time_running(:,:));
ttl1=title( 'Online time synchronization'  )
ttl1=set(ttl1,'Interpreter','latex','FontSize', 15);
xlbl1=xlabel('Iterations [-]');
xlbl1=set(xlbl1,'Interpreter','latex');
ylbl1=ylabel('Time in the node [e.g. seconds,ticks]');
ylbl1=set(ylbl1,'Interpreter','latex');
grid;
% 
figure;
h = plot(G);
c = h.EdgeColor;
h.EdgeColor = 'k';
h.LineWidth=1
h.MarkerSize=8;
h.EdgeAlpha=1;
labelnode(h,1:nodes,' ')
\endtt
\app Content of CD


