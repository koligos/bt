
\def\ctustyle{{\tenss CTUstyle}}
\def\ttb{\tt\char'\\} % pro tisk kontrolních sekvencí v tabulkách

\chap Introduction

Topic of distributed algorithms used in radio communication has been for last decades subject of research. In this thesis we are studying  and commenting the basic results found in the corresponding literature, with emphasis on the understanding of the average consensus algorithm.

\sec Motivation

To begin with an idea of the average consensus algorithm, let's make a thought experiment. We are looking for the average quantity, for example an average temperature in a room, with a group of wireless communication devices, that can exchange information, providing they are in range to reach each other. We deploy these thermometers in the room randomly, with no special requirements on a topology. We need to consider, that for each pair of the thermometers we could determine, whether they can exchange information or not -- meaning we are able to get know all neighbors of all devices, that are mutually in range to communicate.

Now, we can encode our experiment settings to a graph. Drawing this graph is  quite natural way to represent it. In order to do so, we simply take all thermometers as different vertices. Indubitably, every vertex always knows a result of its own measurement. By an edge between two vertices we mark the relation, that these two nodes can exchange information. Which means, each node can get know also the value that measures its neighbor. This ought to be only a basic illustration how to transfer this physically realizable experiment to the terms of graphs.

Finally, as we shall see, if we fulfill some basic convergence conditions on the properties of this graph, the average consensus algorithm should act like this: We synchronously update the value in each node by some increment, depending only on the old value in this node and the values of its direct neighbors in the graph. By doing this long enough, we are going to obtain a new value for each node, which is in limit going to the average of all initially measured values.

\sec Outline

%In practise is sometimes extremely useful to use graphs to represent a given problem. Totally different problems may be surprisingly found to have common and simple solution when represented as the graph.

Graph theory provides an elegant way to represent information encoded by graphs as matrices. In the first chapter we will provide some basic definitions to the Graph theory and properties of these important matrices. Using matrices, we will also briefly mention some  useful results from Matrix analysis, because a serious object of our interest will be topic of eigenvalues of matrices. We will define a Laplacian of a graph and show some of its basic properties. In the following chapter, we will provide a description of the average consensus algorithm and show some examples with graphically illustrated solution. In the last part of this thesis, we will try to implement this algorithm, so we will be able to solve some typical basic problems in area of wireless digital communication, e.g. time base synchronization problem. In a basic case, we can also show, how do the nonidealities change the result of algorithm, in our case additive zero mean noise.


