% The documentation of the usage of CTUstyle -- the template for
% typessetting thesis by plain\TeX at CTU in Prague
% ---------------------------------------------------------------------
% Petr Olsak  Jan. 2013

% You can copy this file to your own file and do some changes.
% Then you can run:  pdfcsplain your-file

\input ctustyle   % The template is included here.
%\input pdfuni    % Uncomment this if you need accented PDFoutlines
%\input opmac-bib % Uncomment this for direct reading of .bib database files 

\worktype [B/EN] % Type: B = bachelor, M = master, D = Ph.D., O = other
                 % / the language: CZ = Czech, SK = Slovak, EN = English

\faculty    {F3}  % Type your faculty F1, F2, F3, etc.
            % use main language of your document here:
\department {Department of Radioelectronics}
\title      {Distributed signal processing in radio communication networks}
%\subtitle   {}
            % \subtitle is optional
\author     {Jakub Kolář}
\date       {November 2016}
\supervisor {}  % One or more supervisors
\studyinfo  {Open Electronic Systems}  % Study programme etc.
\workname   {Bachelot thesis} % Used only if \worktype [O/*] (Other)
            % optional more information about the document:
\workinfo   {kolarj39@fel.cvut.cz}
            % Title / Subtitle in minor language:
\titleEN    {CTUstyle -- the user manual}
\subtitleEN {the plain\TeX{} template for theses at CTU}
            % If minor language is other than English
            % use \titleCZ, \subtitleCZ or \titleSK, \subtitleSK instead it.
\pagetwo    {}  % The text printed on the page 2 at the bottom.

\abstractEN {
   Fireflyes are really amazing.
}
\abstractCZ {
   Pokus napsat bakalarku a take ji obhajit.
}          

\keywordsEN {   averaging consensus algorithm
}
\keywordsCZ {%
   konsensus algoritmus
}
\thanks {           % Use main language here
  Thank you all.
}
\declaration {      % Use main language here
 I announce I made this thing alone.
  In Prague, 3. 12. 2016 % !!! Attention, you have to change this item.
   \signature % makes dots
}

%%%%% <--   % The place for your own macros is here.

%\draft     % Uncomment this if the version of your document is working only.
%\linespacing=1.7  % uncomment this if you need more spaces between lines
                   % Warning: this works only when \draft is activated!
%\savetoner        % Turns off the lightBlue backround of tables and
                   % verbatims, only for \draft version.
%\blackwhite       % Use this if you need really Black+White thesis.
%\onesideprinting  % Use this if you really don't use duplex printing. 

\makefront  % Mandatory command. Makes title page, acknowledgment, contents etc.

\input introduction    % Files where the source of the document is prepared.

\bye
