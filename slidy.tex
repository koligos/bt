\input ctuslides

\worktype[B/EN]
\faculty{F3}
\department{Department of Radioelectronics}

\slideshow 

\tit Distributed signal processing \nl in radio communication

\subtit\bf Jakub Kolář

\subtit\rm Bachelors's thesis presentation

\pg; %------------------------------------------------------------------


\sec Average consensus algorithm \nl on the graph: motivation (1)

* Graph $G$ is a set of edges and vertices
* An existing edge represents a possibility to exchange information
* Initialize $i-$th vertex with value $i$ (e.g. temperature)
$$x_i(0)=i \eqmark$$
* Distributed algorithm: communication is possible only between \nl the neighbors  in the graph
* Goal: obtain unique value in all vertices 
\medskip
\centerline{\picw=9cm \inspic pictures/slides/test_graph.pdf } 
\caption/f { An example of a graph to present the problem }
%\pg+

\pg;

\sec Average consensus algorithm \nl on the graph: motivation (2)


\medskip
\centerline{\picw=12cm \inspic pictures/slides/prubeh_algoritmu.pdf } 
\caption/f { Run of the  Average consensus algorithm on the graph }
\pg;

\sec Graph theory

* Define basic terms with focus on matrix representation of the graph $G$
\begitems
*   Number of nodes $N$
*   Adjacency matrix ${\bi A}(G) \in {\bbchar R}^{N \times {N}} $
*   Degree matrix  ${\bi D}(G) \in {\bbchar R}^{N\times{N}} $
\enditems

* Laplacian of the graph and its spectrum
\begitems
  * Laplacian definition  $${\bi L}(G) = {\bi D}(G) - {\bi A}(G) \eqmark $$
  * The second smallest eigenvalue $\lambda_2$ -- Graph connectivity
\enditems
*  ${\bi L}(G)$ bears relevant information about the graph 
$$ G \Leftrightarrow {\bi L}(G) \eqmark$$

\pg;

\sec Average consensus algorithm \nl on the graph

* Linear update scheme \nl $$ x_i(t+1) = x_i(t)+\sum_{{j \ \in \  Neighbors}} p_{ij}(x_j(t)-x_i(t)) \eqmark $$

* Expressed as matrix multiplication \nl $$ {\bi x}(t+1) = {\bi P} {\bi x}(t) \eqmark $$

* Average and Convergence condition \nl $$ lim_{t\rightarrow\infty} {\bi P}^t = {1\over N}{\bi 1 1}^T \Leftrightarrow  p_{ij} \rightarrow {1\over N} \eqmark $$

* Suits $${\bi P} = {\bi I} - \alpha {\bi L}, \ \alpha < {{1}\over{\Delta}} \in \bbchar R \eqmark $$
\begitems
* $\Delta$ is the greatest degree of the node in the graph
\enditems

\pg;

%to say: az do ted jsem mluvil o necem, co je v literature pomerne presne popsane.
% dale jsem se zabyval problematikou s noisy updates, ktera je stale vyzkumne otevrenym problemem

\sec Noisy updates: problem

* Remains subject of research
* Updates affected by noise  $$ {\bi x}(t+1) = {\bi P} ({\bi x}(t)+{\bi w}(t)) \eqmark$$


\centerline{\picw=0.6\hsize \inspic pictures/slides/noisy_updates.pdf } 
\caption/f Updates affected by zero-mean additive noise


\pg;



\sec Noisy updates: solution


* Decreasing step size \nl $$\alpha \rightarrow \{ \gamma(t) | \gamma(t+1)<\gamma(t)  \}_{t=1}^\infty \eqmark$$


\centerline{\picw=14cm \inspic pictures/slides/conv_noise_values_gamma.pdf } 
\caption/f Run of the algorithm with decreasing step size
\pg;
\sec Noisy updates - solution (2)


* Converges in the sense of decreasing variance

\centerline{\picw=14cm \inspic pictures/slides/conv_noise_var.pdf } 
\caption/f Decreasing variance of values in nodes from the previous example using decreasing step size
\pg;

%\sec Examples of application: \nl dynamic target tracking

%* Damaged version of the algorithm (decreased $\alpha$)
%* Robustness

%\vskip 1cm

%\picw=7.5cm
%\centerline {\inspic pictures/tracking/initialization.pdf \hfil \inspic pictures/tracking/10_iterations.pdf }\nobreak
%\caption/f Initialization and the 10th iteration of the Target tracking simulation.
%\pg;

%\sec Examples of application: \nl dynamic target tracking (2)

%* Target stops moving after 500 iterations 

%\vskip 1cm

%\picw=7.5cm
%\centerline {\inspic pictures/tracking/100_iterations.pdf \hfil \inspic pictures/tracking/1000_iterations.pdf }\nobreak
%\caption/f The 10th and 1000th iteration of  the Target tracking simulation
%\pg;


\sec Example of application: \nl  Distributed time base synchronization (1)

%* The time is represented by some big integer
%* In the simulation increases according to internal oscillator (non ideal) 

%\centerline{\picw=15cm \inspic pictures/slides/time_sync_beg.pdf } 
%\caption/f Run of the Time base synchronization example

* Motivation: we want to use TDMA
* Nodes need to have mutually synchronized time base



\centerline{\picw=11cm \inspic pictures/TDMA_bad.pdf } 
\centerline{\picw=11cm \inspic pictures/TDMA_good.pdf } 
\caption/f Pictures to explain Time Division Multiple Access

\pg;






\sec Example of application: \nl  Distributed time base synchronization  (2)

* Each vertex transmits an impulse at defined moment (e.g. maximum of amplitude)
* We can apply the Average consensus algorithm  on the offsets of impulses received from neighbors (see Figure bellow) 


\centerline{\picw=\hsize \inspic pictures/time_differences.pdf } 
\caption/f Figure to explain detection of time offsets

\pg;





\sec Example of application: \nl  Distributed time base synchronization (3)

* Modified update equation
$$ t_i(n+1) = t_i(n)+T_i+ \sum_{j \ \in  \ Neighbors} p_{ij}(t_j(n) - t_i(n)), \eqmark$$

\centerline{\picw=0.7\hsize \inspic pictures/time_sym.pdf }

\caption/f A run of the Time base synchronization example




\pg;

%\sec Example of application: \nl  Distributed time base synchronization (4)


%\centerline{\picw=\hsize \inspic pictures/PLL.pdf } 
%\caption/f A block diagram representation of solution

%\pg;


%to do priklad jak detekuji typrichazejici impulsy






\sec Conclusion \& My Contribution

I have 

* studied the Linear average consensus algorithm on the graph 
* implemented the  algorithm in  several examples
* observed, that the experimentally obtained outputs are consistent with the previous theoretical part
\pg;

\null
\vskip2cm
\centerline{\typosize[35/40]\bf Thank you for your attention.}\pg;

%\vskip2cm
%\centerline{\Blue\typosize[60/70]\bf Questions?}


\sec  Otázka

*Je možné využít analytického aparátu pro zpracování signálu, viz [1],  \nl   nad grafem při
analýze chování distribuovaných algoritmů?

* Ano, ale zatím spíše jen velmi omezeně [1]: 
\begitems
* Operace pro zpracování signálu nad grafy (harmonická analýza) mohou být zobecněny zavedením příslušných operátorů
* Problém s nepravidelnostmi grafů
\begitems
* Jak graf sestavit v souladu s příslušnou transformací
* Jak využít současné znalosti zpracování signálu
* Výpočetně efektivní řešení 
\enditems
* Nejednoznačný a heuristický přístup (normalizace Laplaciánu)
* Analýza (často) vyžaduje určení vlastních vektorů Laplaciánu
\enditems
* Jedná se o výzkumně otevřený problém [1]

\pg;

\sec References

 [1] David I. Shuman, Sunil K. Narang, Pascal Frossard, Antonio Ortega, and Pierre
Vandergheynst. Signal Processing on Graphs: Extending High-Dimensional
Data Analysis to Networks and Other Irregular Data Domains. CoRR. 2012,
abs/1211.0053 


\pg. %------------------------------KONEC-------------------------------

