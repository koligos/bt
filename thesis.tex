% The documentation of the usage of CTUstyle -- the template for
% typessetting thesis by plain\TeX at CTU in Prague
% ---------------------------------------------------------------------
% Petr Olsak  Jan. 2013

% You can copy this file to your own file and do some changes.
% Then you can run:  pdfcsplain your-file

\input ctustyle  
\def\thednum{(\the\chapnum.\the\dnum)}
 % The template is included here.
%\input pdfuni    % Uncomment this if you need accented PDFoutlines
%\input opmac-bib % Uncomment this for direct reading of .bib database files 

\worktype [B/EN] % Type: B = bachelor, M = master, D = Ph.D., O = other
                 % / the language: CZ = Czech, SK = Slovak, EN = English

\faculty    {F3}  % Type your faculty F1, F2, F3, etc.
            % use main language of your document here:
\department {Department of Radioelectronics}
\title      {Distributed signal processing in~radio communication networks}
%\subtitle   {}
            % \subtitle is optional
\author     {Jakub Kolář}
\date       {May 2017}
\supervisor {prof. Ing. Jan Sýkora, CSc.}  % One or more supervisors
\studyinfo  {Open Electronic Systems}  % Study programme etc.
\workname   {Bachelor's thesis} % Used only if \worktype [O/*] (Other)
            % optional more information about the document:
\workinfo   {kolarj39@fel.cvut.cz}
            % Title / Subtitle in minor language:
\titleEN    {CTUstyle -- the user manual}
\subtitleEN {the plain\TeX{} template for theses at CTU}
            % If minor language is other than English
            % use \titleCZ, \subtitleCZ or \titleSK, \subtitleSK instead it.
\pagetwo    {}  % The text printed on the page 2 at the bottom.

\abstractEN { Main scope of this Bachelor’s thesis
is to provide an introduction to Linear
average consensus algorithm over
graph. It is an iterative algorithm,
that works with values assigned to
vertices of graph and the goal is to
asymptotically obtain in all vertices an
average of initial values, respecting the
graph topology. In the text we use a
matrix representation of graph and to
design the parameters of algorithm is
used Laplacian matrix. Using matrix
representation, the algorithm solving
the original problem is transferred to
iterative matrix multiplication. Next,
we provide examples of algorithm run
in case of ideal and reliable communication and also an outline of solution
in case of zero-mean noisy updates. A
part of work are also Matlab scripts
with algorithm implementation. More
difficult formal proofs and derivations
are not included, but we provide a
reference, where to find them. Finally, we provide few simple examples of application of the described algorithm.
}

\abstractCZ { Cílem této bakalářské práce je seznámení se s lineárním konsenzuálním algoritmem nad grafy. Jde o iterativní algoritmus, který pracuje s hodnotami přiřazenými vrcholům grafu a jehož cílem je, aby asymptoticky byly tyto hodnoty ve všech vrcholech stejné a průměrem hodnot počátečních. Studovaný algoritmus toho docílí jen komunikací mezi dvojicemi vrcholů přes existující hrany grafu.  V práci využíváme maticovou reprezentaci grafu a pro nalezení vhodných parametrů algoritmu používáme Laplaceovu matici. Algoritmus je tak převeden na opakující se maticové násobení, jehož význam zachovává původní úlohu. Dále uvádíme  názorné ukázky toho, jak algoritmus pracuje v případě ideálních aktualizací a základní řešení v případě ovlivňování aktualizací šumem s nulovou střední hodnotou. Součástí práce je implementace algoritmu v Matlabu. V práci nejsou uvedeny složitější důkazy a odvození, ale odkazujeme se na příslušnou uvedenou literaturu. Na závěr uvádíme několik jednoduchých příkladů využívajících popsané iterativní schéma.
}          

\keywordsEN {       Graph, Laplacian, Linear average consensus algorithm, Noisy updates, Estimation

}
\keywordsCZ {%
   Graf, Laplaceova matice, Lineární konsenzus, Aktualizace s rušením, Odhad parametru
}
\thanks {           % Use main language here
  % The first great thank belongs to prof. Ing. Jan Sýkora, CSc., for such an interesting assignement offer and precious advice during the consultations, too. Also I'd like to write here thanks to  all my  beloved friends and  family, who  all formed me, inspire me and motivate me as well.
}
\declaration {      % Use main language here
   TBD.
  In Prague, 26. 5. 2017 % !!! Attention, you have to change this item.
   \signature % makes dots
}
\specification {\picw=\hsize \cinspic appendix/assignment.pdf}


%%%%% <--   % The place for your own macros is here.

%\draft     % Uncomment this if the version of your document is working only.
%\linespacing=1.7  % uncomment this if you need more spaces between lines
                   % Warning: this works only when \draft is activated!
%\savetoner        % Turns off the lightBlue backround of tables and
                   % verbatims, only for \draft version.
%\blackwhite       % Use this if you need really Black+White thesis.
%\onesideprinting  % Use this if you really don't use duplex printing. 

\makefront  % Mandatory command. Makes title page, acknowledgment, contents etc.
\input opmac-bib
\input introduction  % Files where the source of the document is prepared.
\input graph_theory
\input consensus_problem
\input distributed_estimation_in_wirelless_network
\input prilohy



\bye
