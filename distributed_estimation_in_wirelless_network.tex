\def\ctustyle{{\tenss CTUstyle}}
\def\ttb{\tt\char`\\} % pro tisk kontrolních sekvencí v tabulkách

\chap Distributed Estimation in Wireless Sensor Networks

The following chapter should serve as an description of the problematics of distributed estimations in wirelless networks. We will summarize firstly the problematics overview, with focus on the aspects that complicate the estimation process,  and   then provide description of some basic approaches of implementation of a consensus algorithm, that respects the true noniedeal properties of real networks. 

This chapter is based on source \cite[spring_book].

\sec{Introduction}

Wireless networks are present in a great number of applications of an everyday life. The most commonly given examples are eg. communication systems (mobile networks, LTE, Wi-Fi), all sorts of measuring devices (technical, medical, industry and other usage). Currently very popular are autonomous cars. In the context of our main topic, the averaging consensus algorithm might be commonly seen in solving a problem of moving in coordinated formations (eg. flights of drons and others unmaned vehicles).

Subject of our interest will be now Wirelless Sensor Networks (WSN) with smart nodes, i.e. nodes, that own some computing power and are programmable (i.e. not just an ordinary sensor). By localizing such nodes in some area to create a network, we aim to  avoid using any kind of centralized topology. One benefit is, that WSN becomes to be much more robust with respect to a single failure point of a central device. (We do not have any). Naturally, now we are not limited to a fixed topology and the nodes can move. Even more important is, that in a situation, when nodes of WSN aim to  estimate some value $ \theta$ in a distributed manner, it is naturally much more faster when the nodes communicate directly. Having nodes, that are battery-supplied, we are also glad to avoid using the limited amount of power without necessity. It seems naturally, that more effective way is to use only direct communication between nodes, that want to obtain the desired estimation, without employing some distant data center. 

WSN networks may consist of hundreds of nodes. Its goal is to estimate   the value cooperatively and locally, instead of moving the calculation somewhere else, beacuase sometimes it might not be even possible. Typically, in wireless communication the nodes should obtain an overview of the network parameters (network topology, carrier frequency, common time base). Concerning the topology, one node initially knows only its direct neighbors. In digital communication applications, is typically necessary, that each node knows the topology of whole network (an Adjacency matrix). This is exactly the case, when we can not avoid a cooperation between nodes. With respect to the previous chapters, we note that the WSN can be for a purpose of distributed consensus algorithm simply coded into a graph with, expressing the possibility of an information exchange between nodes \cite[spring_book]. 



\sec Overview of Distributed Consensus Estimation

Our general observation model will be
$$ z= {\bi H}x+w, \eqmark$$
where $z$ is our observation, $x$ is the target value,  ${\bi H}$ is an observation matrix and $w$ is the additive noise. 

Typical problems in WSN estimation process are:
\begitems \style 0
\item{*}Noise in the network. We generally never receive the same value as we send. The observation process is also affected by noise.
\vskip 0.2cm
\item{*}Wireless devices are battery-powered and because of that, the estimation process should be effective and not to waste the power.
\vskip 0.2cm
\item{*}Topology of WSN may generally change.
\vskip 0.2cm
\item{*}The algorithm requires a common clock synchronization between nodes.
\vskip 0.2cm
\item{*}The communication between two nodes generally must not have symmetric characteristics, so that e.g. $A$ node can send information to $B,$ however $B$ is not able to $A.$  Hence, a graph describing the WSN inter-nodes communication is then the directed graph. 
\enditems
\vskip 0.2cm


\secc Consensus-Based Distributed Parameter Estimation 

In \cite[spring_book] may be found quite general concept of solution to the estimation process in WSN. We will use it to present the main difficulties related to the algorithm design. The authors there use two kinds of sensors: Sensor Nodes (SN), that locally measure the value we want to globally estimate and  Relay Nodes (RN), that do not measure anything but only distribute the measure results of SN to other nodes. This concept would have been practically used to save money for many expensive sensors. Such a network we call heterogeneous and hence, all the nodes do not have same impact on the result. By adding RN into the WSN, we  also naturally increase the connectivity of the graph. Typically, we can add cheap RN into the network to enable and/or improve the connectivity of all SN. 

Next we briefly list the main difficulties that must be considered using this model.
 
\secc Asymetric communication

Since this model describes situation with asymmetric communication channels a directed graph representation must be used.  In our case, a weigted  graph $G$ is a triplet $G= ( V, E, {\bi A})$, where $V$ is the set of RN and SN,  $ E \subset V\times V,$ and ${\bi A}$ is a matrix of weights associated to edges $E.$ For elements of $\bi A$ holds $$a_{ij} > 0 \Leftrightarrow (i,j)\in E. \eqmark$$ Note, this $\bi A$ matrix is a generalization of already before used Adjacency matrix. Considering directed graphps, it is in a litereature common to call a source vertex of an edge {\em parent} and the destination vertex {\em child }\cite[spring_book].

Assuming, that $i-$th node transmits to the $j-$th at a constant power level $P_{T_i}$ in distance $d_{ij}$, the communication will be successful if holds the  inequality for Signal to Noise Ratio (SNR)
\label[snr]$$  {{P_{T_i}}\over{{\bbchar N}d_{ij}^\eta}} \ge \beta  , \eqmark$$
where $  {\bbchar N} $ stands for a power level of noise in the channel, $\eta$ is an exponent expressing the lossy behavior of the channel and $\beta$ is the minimum SNR value fulfilling the condition for communication \cite[spring_book]. 
From Equation \ref[snr] we can determine a maximum distance $d_{ij_{\max}}$ between nodes $i, j,$ that will serve as a threshold for evaluating the communication as possible
$$ d_{ij_{\max}}= \root {\eta}\of{{P_{T_i}}\over{{\bbchar N}\beta}}. \eqmark$$ 


\secc Multidimensional observation

\cite[spring_book] solves a problem of estimation of a vector $ \theta \in {\bbchar R}^J,$ whose components are separately measured by SN. The measurement of the desired vector $ \theta$ is described as

\label[observation] $$ y_i(t)={\bi H_i}  \theta + w_i(t), \forall i \in I_S  \eqmark$$

where ${\bi H_i} \in {\bbchar R}^{J_i \times J}$ is an observation matrix and $w_i(t)$ is white Additive Gaussian noise. The statement  $J_i\le J$ for $i-$th SN means, that it generally provides only limited information about the vector $ \theta,$ because it can't measure the rest of components. The RN nodes can't measure $ \theta$ at all, because they are not equiped with the sensors.

The vector nature of $  \theta$ will consequently bring to the matrix description of the update equations, similiar to the Equation \ref[ue] Kronecker product. Although such a description is possible, it is very confusing and  is reasonable to use probably only in theory. To see this matrix description, we refer to \cite[spring_book].    

\Definition (Kronecker Product) \cite[kroneckerPrd] Given matrix ${\bi A}\in {\bbchar R}^{m\times n}$  and matrix ${\bi B}\in {\bbchar R}^{p\times q},$ their Kronecker Pruduct ${\bi C}\in {\bbchar R}^{mp\times nq}$ is denoted
$${\bi C}= {\bi A }\otimes{\bi B } ,$$
where $c_{\alpha \beta}= a_{ij}b_{kl},$ using $\alpha=p(i-1)+k$ and $\beta=q(j-1)+l.$ 
To give an intuition, we provide a simple example. 

\Example (Kronecker product practise) Calculate a Kronecker product ${\bi C}= {\bi A }\otimes{\bi J },$ where
$$ {\bi A}= \pmatrix{a&b\cr c&d}, {\bi J}= \pmatrix{j&k \cr l&m}.$$

{\em Solution:}

 $$ {\bi C}=  {\bi A }\otimes{\bi J } =\pmatrix{aj & ak& bj& bk\cr al & am& bl& bm\cr cj & ck & dj & dk \cr cl & cm & dl & dm }.$$


\secc Description of Algorithm DCUE

The DCUE algorithm assumes perfectly synchronized updates. In each node $i$ is initially known a value $x_i(0).$  $ {\cal{N}}_i^S$ and  ${\cal{N}}_i^R$ marks the set of SN neighbors and RN neighbors to the $i-$th node, respectively. The update equation of DCUE algorithm states:
\label[algorithm_SN] $$ \eqalign{
 x_i(t+1) = x_i(t) +  \rho(t)\alpha_i    {\bi H_i}^T \left(y_i(t) - {\bi H_i}  x_i(t)  \right) 
+ \cr + {\rho(t) \over{c_i}}\left[ \sum_{j\in {\cal{N}}_i^S  } a_{ij} (x_j(t)-x_i(t)) + \sum_{j\in {\cal{N}}_{i}^R  } a_{ij} (z_j(t)-x_i(t))   \right],
} \eqmark$$
where $\alpha_i>0$ controls  the update rate of information during the run of algorithm; $\rho(t)>0$ is a weight controlling an impact of received updates; $c_i>0$ controls impact of $i-$th own measurement and $ a_{ij}=\sqrt{{P_{T_j}|h_{ij}|^2}\over{   d_{ij}^\eta }}$ represents an amplitude of a signal received by node $i$ from node $j$ , in which $h_{ij}$ is a fading coefficient describing a channel between $i$ and $j,$ and it is a reason why $a_{ij} \not=a_{ji}$ \cite[spring_book].

Next, an update equation for RN nodes reads
 \label[algorithm_RN] $$z_i(t) =\sum_{j\in {\cal{N}}_i^S  } \gamma_{ij}x_j(t)+ \sum_{j\in {\cal{N}}_i^R  } \gamma_{ij}z_j(t)   , \forall i \in I_R, \eqmark$$ where $\gamma_{ij}$ are some nonnegative weighting coefficents \cite[spring_book].


Summarizing the meaning of equations  describing the algorithm, that we stated above: Firstly, Equation \ref[observation] says, that observation of node $i$,  \ $y_i(t)$, in time $t$ are elements of vector $\theta$, according to its appropriate matrix $\bi H_i$,  and this observation is affected by the Gaussian noise $w_i.$  Next, Equation \ref[algorithm_SN] proposes the way how $i-$th SN should update the values to be estimated. It consists of a specific linear combination of values that node $i$ self measures and values that it recieves from neighbors SNs and RNs, respectively. Doing this, we take into account distance between neighbors (i.e. in units of length, not number of hops), properties of the channel between nodes, transmitting power and consequently we decide, whether the update can be applied, according to these channel characteristcs. Finally, Equation \ref[algorithm_RN] is an analogy update equation for $i$-th RN, whose updates are determined as a linear combination from values of its RNs and SNs neighbors.














