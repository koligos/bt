\input ctuslides

\worktype[B/EN]
\faculty{F3}
\department{Department of Radioelectronics}

\slideshow 

\tit Distributed signal processing \nl in radio communication

\subtit\bf Jakub Kolář

\subtit\rm Bachelors's thesis presentation

\pg; %------------------------------------------------------------------


\sec Average consensus algorithm \nl on the graph: motivation (1)

* Graph is a set of edges and vertices
* An existing edge represents a possibility to exchange information
* Initialize $i-$th vertex with value $i$ (e.g. temperature)
$$x_i(0)=i \eqmark$$
* Distributed algorithm: communication only between the neighbors in the graph.  
\medskip
\centerline{\picw=9cm \inspic pictures/slides/test_graph.pdf } 
\caption/f { An example of a graph to present the problem }
%\pg+

\pg;

\sec Average consensus algorithm \nl on the graph: motivation (2)


\medskip
\centerline{\picw=12cm \inspic pictures/slides/prubeh_algoritmu.pdf } 
\caption/f { Run of the  Average consensus algorithm on the graph }
\pg;

\sec Graph theory

* Define basic terms with focus on matrix representation of the graph
\begitems
*   Adjacency matrix ${\bi A}(G)$
*   Degree matrix  ${\bi D}(G)$
\enditems

* Laplacian of the graph and its spectrum
\begitems
  * Laplacian definition  $${\bi L}(G) = {\bi D}(G) - {\bi A}(G) \eqmark $$
  * The second smallest eigenvalue $\lambda_2$ -- Graph connectivity
\enditems
*  ${\bi L}(G)$ bears relevant information about the graph 
$$ G \Leftrightarrow {\bi L}(G) \eqmark$$

\pg;

\sec Average consensus algorithm \nl on the graph

* Update scheme \nl $$ x_i(t+1) = x_i(t)+\sum_{j=1; j\not= i}^N p_{ij}(x_j(t)-x_i(t)) \eqmark $$

* Expressed as matrix multiplication \nl $$ {\bi x}(t+1) = {\bi P} {\bi x}(t) \eqmark $$

* Average and Convergence condition \nl $$ lim_{t\rightarrow\infty} {\bi P}^t = {1\over N}{\bi 1 1}^T \Leftrightarrow  p_{ij} \rightarrow {1\over N} \eqmark $$

* Suits $${\bi P} = {\bi I} - \alpha {\bi L}, \ \alpha = {{1}\over{\Delta}} \in \bbchar R \eqmark $$
\begitems
* $\Delta$ is the greatest degree of the node in the graph
\enditems

\pg;



\sec Noisy updates: problem

* Updates affected by noise $$ {\bi x}(t+1) = {\bi P} ({\bi x}(t)+{\bi w}(t)) \eqmark$$


\centerline{\picw=0.6\hsize \inspic pictures/slides/noisy_updates.pdf } 
\caption/f Updates affected by zero-mean additive noise


\pg;



\sec Noisy updates: solution


* Decreasing step size \nl $$\alpha \rightarrow \{ \gamma(t) | \gamma(t+1)<\gamma(t)  \}_{t=1}^\infty \eqmark$$


\centerline{\picw=14cm \inspic pictures/slides/conv_noise_values_gamma.pdf } 
\caption/f Run of the algorithm with decreasing step size
\pg;
\sec Noisy updates - solution (2)


* Converges in the sense of decreasing variance

\centerline{\picw=14cm \inspic pictures/slides/conv_noise_var.pdf } 
\caption/f Decreasing variance from the previous example using decreasing step size
\pg;

%\sec Examples of application: \nl dynamic target tracking

%* Damaged version of the algorithm (decreased $\alpha$)
%* Robustness

%\vskip 1cm

%\picw=7.5cm
%\centerline {\inspic pictures/tracking/initialization.pdf \hfil \inspic pictures/tracking/10_iterations.pdf }\nobreak
%\caption/f Initialization and the 10th iteration of the Target tracking simulation.
%\pg;

%\sec Examples of application: \nl dynamic target tracking (2)

%* Target stops moving after 500 iterations 

%\vskip 1cm

%\picw=7.5cm
%\centerline {\inspic pictures/tracking/100_iterations.pdf \hfil \inspic pictures/tracking/1000_iterations.pdf }\nobreak
%\caption/f The 10th and 1000th iteration of  the Target tracking simulation
%\pg;


\sec Example of application: \nl  Distributed time base synchronization 

* The time is represented by some big integer
* In the simulation increases according to internal oscillator (non ideal) 

\centerline{\picw=15cm \inspic pictures/slides/time_sync_beg.pdf } 
\caption/f Run of the Time base synchronization example



\pg;








\sec Conclusion \& My Contribution

I have 

* studied the Linear average consensus algorithm on the graph 
* implemented the  algorithm in  several examples motivated by the radio digital communication
* observed, that the experimentally obtained outputs are consistent with the previous theoretical part
\pg;

\null
\vskip2cm
\centerline{\typosize[35/40]\bf Thank you for your attention.}\pg+

\vskip2cm
\centerline{\Blue\typosize[60/70]\bf Questions?}

\pg. %------------------------------KONEC-------------------------------

